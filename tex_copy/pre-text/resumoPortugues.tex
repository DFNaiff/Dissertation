\begin{abstract}

%MADUREIRA, Rodrigo Lopes Rangel. Algoritmos de interseções, bla bla bla...\\

Inferência Bayesiana é um dos paradigmas principais da Estatística, um que nos ultimos anos se mostrou bem sucedido. Porém, várias técnicas para ele requer um numero grande de avaliações de \textit{posterioris}, o que pode ser inviável em algumas aplicações. Baseado em trabalhos pré-existentes, um novo algorítmo de inferencia aproximada é desenvolvido neste trabalho, baseado em inferência variacional e processos Gaussianos. Esse algorítmo, nomeado pelo autor Boosted Variational Bayesian Monte Carlo, é desenvolvido para que poucas avaliações de \textit{posteriori} sejam necessárias. O algorítmo, a teoria por trás dele, e seu pacote associado são apresentados, sendo aplicado em um número de exemplos.

\end{abstract}
