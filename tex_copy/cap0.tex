\thispagestyle{empty} \begin{center} {\bf {\Large Universidade
Federal do Rio de Janeiro}}
\bigskip
\bigskip
\bigskip
\bigskip
\bigskip

{\bf {\LARGE{\emph {Boosted Variational Inference via Bayesian Monte Carlo}}}}
\bigskip
\bigskip
\bigskip
\bigskip

{\bf{Danilo de Freitas Naiff}}

\bigskip
\bigskip
\bigskip
\bigskip
\bigskip
\bigskip
\bigskip
\bigskip
\bigskip
\bigskip
\bigskip
\bigskip
\bigskip
\bigskip
\bigskip
\bigskip
\bigskip
\bigskip
\bigskip
\bigskip
\bigskip
\bigskip
\bigskip
\bigskip
\bigskip
\bigskip
\bigskip
\bigskip
\bigskip
\bigskip
\bigskip
\bigskip
\bigskip
\bigskip
\bigskip
\bigskip
\bigskip
\bigskip
\bigskip

{\large \bf {Rio de Janeiro}}

\bigskip

{\large \bf{Setembro de 2019}}
\end{center}

\newpage
%\thispagestyle{empty}
\pagenumbering{roman}

\begin{center}
{\bf {\Large Universidade Federal do Rio de Janeiro}}
\end{center}
\bigskip
\bigskip

\begin{center}
 {\Large{\emph {Boosted Variational Inference via Bayesian Monte Carlo}}}
\bigskip
\bigskip
\\
{Danilo de Freitas Naiff}
\end{center}

\bigskip
\bigskip
\bigskip

Disserta{\c c}\~{a}o de Mestrado apresentada ao Programa de P{\'o}s-gradua{\c c}\~{a}o
em Matem{\'a}tica, Instituto de Matem{\'a}tica, da Universidade
Federal do Rio de Janeiro (UFRJ), como parte dos requisitos
necess{\'a}rios {\`a} obten{\c c}\~{a}o do t{\'i}tulo de Mestre em Matem{\'a}tica.

\bigskip

\noindent Orientador: Prof. Fabio Ant\^onio Tavares Ramos.

\bigskip
\bigskip
\bigskip
\bigskip
\bigskip
\bigskip
\bigskip
\bigskip
\bigskip
\bigskip
\bigskip
\bigskip
\bigskip
\bigskip
\bigskip
\bigskip
\bigskip
\bigskip
\bigskip
\bigskip
\bigskip
\bigskip
\bigskip
\bigskip
\bigskip
\bigskip
\bigskip
\bigskip
\bigskip


\begin{center}
{\bf  Rio de Janeiro}\\
{\bf  Setembro de 2019}
\end{center}

\newpage
%\thispagestyle{empty}

\begin{center}
{\bf {\Large Universidade Federal do Rio de Janeiro}}
\end{center}
\bigskip
\bigskip

\begin{center}
 {\Large{\emph {Boosted Variational Inference via Bayesian Monte Carlo}}}
\bigskip
\bigskip
\\
{Danilo de Freitas Naiff}
\end{center}

\bigskip
\bigskip
\bigskip

Disserta{\c c}\~{a}o de Mestrado apresentada ao Programa de P{\'o}s-gradua{\c c}\~{a}o
em Matem{\'a}tica, Instituto de Matem{\'a}tica, da Universidade
Federal do Rio de Janeiro (UFRJ), como parte dos requisitos
necess{\'a}rios {\`a} obten{\c c}\~{a}o do t{\'i}tulo de Mestre em Matem{\'a}tica.

{\indent{Aprovada por:}}
\ \\
\ \\

{\indent{-----------------------------------------------------------}}\\
{\indent{Prof. Fabio Ant\^onio Tavares Ramos}\\
\ \\

{\indent{-----------------------------------------------------------}}\\
{\indent{Prof. Heudson Tosta Mirandola} \\
	\ \\
	
{\indent{-----------------------------------------------------------}}\\
{\indent{Prof. Yuri Fahham Saporito} \\
\ \\

{\indent{-----------------------------------------------------------}}\\
{\indent{Prof. Amaro Gomes Barreto Junior} \\
\ \\

{\indent{-----------------------------------------------------------}}\\
{\indent{Prof. Hugo Tremonte de Carvalho} \\
	\ \\

{\indent{-----------------------------------------------------------}}\\
{\indent{Prof. Adriano Mauricio de Almeida Cortes} \\
\ \\



\bigskip
\bigskip
\bigskip
\bigskip
\bigskip

\begin{center}
{\bf Rio de Janeiro}\\
{\bf Setembro de 2019}
\end{center}

\newpage

\begin{tabular}{|p{12cm}|}
\hline
\ \\
\hspace{1cm}Naiff, Danilo de Freitas.\\
\hspace{1.6cm}Boosted Variational Inference via Bayesian Monte\\\hspace{1cm}Carlo/\;Danilo de Freitas Naiff. - Rio de Janeiro:\\\hspace{1cm}UFRJ/\;IM,\;2019.\\
\hspace{1.6cm}xiii,135f.: il.; 29,7 cm.\\
\hspace{1.6cm}Orientador: Fabio Ant\^onio Tavares Ramos.\\
\hspace{1.6cm}Disserta{\c c}\~{a}o (mestrado)\,---\,Universidade Federal do \\ \hspace{1cm}Rio de Janeiro, Instituto de Matem\'atica, Programa de\\ \hspace{1cm}P\'os-Gradua{\c c}\~{a}o em Matem\'atica, 2019.\\
%\hspace{1.6cm}Refer�ncias %Bibliogr�ficas: p. 165-184 .\\
\hspace{1.6cm}Refer\^encias Bibliogr\'aficas: f. 101-114. \\
 \\
\hspace{1.6cm}1. Bayesian Inference. 2. Variational Inference. \\\hspace{1cm}3. Gaussian Processes. 4.\;Bayesian Monte Carlo. I. Ramos, \\\hspace{1cm}Fabio Ant\^onio Tavares. II. Universidade Federal do Rio de\\\hspace{1cm}Janeiro,Programa de P\'os-Gradua{\c c}\~{a}o em Matem\'atica. III.\\\hspace{1cm}T\'itulo.\\
\hline
\end{tabular}

\newpage

\ \\
\vspace{4cm}

\begin{center}
Dedicado a Ronaldo Lemos Naiff.\\
\end{center}

\newpage

\begin{center}
{\bf{\large Acknowledgments}}
\end{center}
\bigskip

\textit{The following acknowledgments are written in portuguese}

\bigskip
Agrade{\c c}o ao meu orientador F\'abio, que me deu o apoio e conhecimento necess\'ario para desenvolver esse trabalho. Agrade{\c c}o ao pessoal do LABMA, pelo companheirismo e pelas ideias que me deram. Agrade{\c c}o aos meus colegas e \`a equipe do FMTC, competi{\c c}\~ao que me ensinou muitas coisas, algumas usadas neste trabalho. De forma mais abstrata, existe toda uma comunidade que permite que id\'eias possam ser compartilhadas, e implementadas, com uma facilidade muito maior do que se teria em d\'ecadas passadas, e eu agrade{\c c}o a ela.

Agrade{\c c}o a todos os amigos que me deram suporte por todos esses anos. Em particular, aqueles que me ouviram falando sobre meus estresses durante a escrita deste texto, sobre problemas em que seus pontos de vista eram absurdamente abstratos. Agrade{\c c}o a minha m\~ae Rose, pelo suporte durante esse tempo, assim como ao meu av\^o Reinaldo. Finalmente, agrade{\c c}o ao meu pai Ronaldo, que nunca imaginaria que eu acabaria por fazer um mestrado em Matem\'atica, mas que acho que n\~ao ficaria triste com isso.

Finalmente, este \'e um trabalho financiado por ag\^encias p\'ublicas, o que no final \'e dinheiro de um povo sofrido, que em sua maioria ganha muito menos do que o que me foi oferecido. Agrade{\c c}o a este, e espero que o ganho total que possa dar de volta justifique este investimento.



\begin{flushright}
Danilo de Freitas Naiff\\
Setembro de 2019
\end{flushright}

\newpage
\begin{center}
{\bf {\large Resumo}}
\end{center}
\bigskip


\begin{center}
 {\large{\emph {Boosted Variational Inference via Bayesian Monte Carlo\\}}}
\bigskip
\bigskip
{Danilo de Freitas Naiff}
\end{center}

\bigskip

Resumo da disserta{\c c}\~{a}o de Mestrado apresentada ao Programa de P{\'o}s-gradua{\c c}\~{a}o
em Matem{\'a}tica, Instituto de Matem{\'a}tica, da Universidade
Federal do Rio de Janeiro (UFRJ), como parte dos requisitos
necess{\'a}rios {\`a} obten{\c c}\~{a}o do t{\'i}tulo de Mestre em Matem{\'a}tica.


\bigskip

\begin{quote}
{\bf \emph{Resumo:}} A maioria dos problemas importantes em aprendizado de m\'aquina s\~ao caros computacionalmente, em particular aqueles que envolvem infer\^encia Bayesiana. Muitas das t\'ecnicas Bayesianas atuais requerem um alto n\'umero de avalia{\c c}\~oes de verossimilhan{\c c}a, o que pode ser invi\'avel em alguns casos. Nessa disserta{\c c}\~ao, propomos um algoritmo de infer\^encia aproximada, baseado em trabalhos recentes sobre infer\^encia variacional e processos Gaussianos. Este algor\'itmo, nomeado pelo autor \textit{Boosted Variational Bayesian Monte Carlo}, \'e construido de forma que poucas avalia{\c c}\~oes de verossimilhan{\c c}a sejam necess\'arias. O algoritmo, seu pacote associado, e a teoria por tr\'as dele s\~ao apresentados neste trabalho. Tambem ilustramos o algoritmo atrav\'es de sua implementa{\c c}\~ao em alguns \textit{toy examples}, e em um problema de fonte de contamina{\c c}\~ao.

\end{quote}



\bigskip

\noindent{\bf{Palavras--chave.}} Infer\^encia Bayesiana, Infer\^encia variacional, Processos gaussianos, Bayesian Monte Carlo.

\bigskip
\bigskip
\bigskip
\bigskip
\bigskip
\bigskip
\bigskip
\bigskip
\bigskip
\bigskip
\bigskip

\ \\
\begin{center}
{\bf \large Rio de Janeiro}\\
{\bf \large Setembro de 2019}
\end{center}

\newpage

\begin{center}
{\bf {\large {\emph {Abstract}}}}
\end{center}
\bigskip

\begin{center}
 {\large{\emph {Boosted Variational Inference via Bayesian Monte Carlo\\}}}
\bigskip
\bigskip
{Danilo de Freitas Naiff}
\end{center}

\bigskip

{\emph {Abstract}} da disserta{\c c}\~{a}o de Mestrado apresentada ao Programa de P{\'o}s-gradua{\c c}\~{a}o
em Matem{\'a}tica, Instituto de Matem{\'a}tica, da Universidade
Federal do Rio de Janeiro (UFRJ), como parte dos requisitos
necess{\'a}rios {\`a} obten{\c c}\~{a}o do t{\'i}tulo de Mestre em Matem{\'a}tica.

\bigskip

\begin{quote}
{\bf \emph{Abstract:}} Most of the important problems in machine learning are computationally expensive, in particular the ones involving Bayesian inference. Many of the current Bayesian techniques requires a large number of likelihood evaluations, which may be infeasible for some cases. In this dissertation, we propose an approximate inference algorithm, based on recent works on variational inference and Gaussian processes. This algorithm, named by the author \textit{Boosted Variational Bayesian Monte Carlo}, is designed so that few likelihood evaluations are needed. The algorithm, its associated package, and the theory behind it are presented in this work. We also illustrate the algorithm by implementing it in some toy examples, and in a contamination source problem.
\end{quote}

\bigskip

\noindent{\bf{Keywords.}} Bayesian inference, variational inference, Gaussian processes, Bayesian Monte Carlo.


\bigskip
\bigskip
\bigskip
\bigskip
\bigskip
\bigskip
\bigskip
\bigskip
\bigskip
\bigskip
\bigskip


\ \\
\begin{center}
{\bf \large Rio de Janeiro}\\
{\bf \large September 2019}
\end{center}